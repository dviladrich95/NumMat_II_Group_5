\newcommand{\assignmentDate}{November 18th, 2019}

% Add title
%Institute
\begin{tabular*}{\hsize}{l@{\extracolsep{\fill}} r}
	\textsc{Technical University of Berlin}		 \hfill&								 	\\
	Faculty II - Mathematics and Natural Sciences\hfill&									\\
	Institute of Mathematics 					 \hfill&									\\
	Dr. D. Peschka, A. Selahi 		 			 \hfill&									\\
\end{tabular*}

% Title
\begin{center}
	\textbf{\Large{\courseName}}\\
	\vspace{7pt}
	\large{Homework \currentAssignment}\\
	\smallskip
	\normalsize{Submitted on \assignmentDate}
\end{center}

% Group table
\begin{center}
	\vspace{-8pt}
	\begin{tabular}{l c r}
		by \textbf{\groupNumber}		    &	 			  &		 								\\
		\hline
		\texttt{Kagan Atci} 			    & \texttt{338131} & \texttt{Physical Engineering, M.Sc.}\\
		\texttt{Navneet Singh }		 	    & \texttt{380443} & \texttt{Scientific Computing, M.Sc.}\\
		\texttt{Riccardo Parise }		    & \texttt{412524} & \texttt{Scientific Computing, M.Sc.}\\
		\texttt{Daniel V. Herrmannsdoerfer} & \texttt{412543} & \texttt{Scientific Computing, M.Sc.}\\ 
		\hline
	\end{tabular}
\end{center}

% EXERCISE 1
% --------------------------------------------------------------------------------------------------------------------
\addExercise{1}{Ex1}
Considered is the Laplace operator $Lu = -\Delta$ in $\Omega = (0, L)^2$ and $u \in C^4(\bar{\Omega})$.
%
% ----------------
\newcommand{\bigO}[1]{\mathcal{O}(#1)}
\newcommand{\fixDot}{\;\cdot\;}
\addSubExercise{a}
Let $\VECT{x} = [x_1, x_2]^T \in \Omega_h$.
Applying the Laplace operator in the form of finite difference on $u(\VECT{x})$ takes place by differentiating $u(\VECT{x})$ with respect to $x_1$ and $x_2$ one at a time.
The $D^-D^+$ stencil is employed as the 2nd order finite difference operator with one neighbor on either side at the distance $h$.
\begin{align}
	\label{eq:diffX1}
	\frac{\partial^2u}{\partial x_1 ^2} = \frac{u(x_1 + h, x_2) - 2u(x_1, x_2) + u(x_1 - h, x_2)}{h^2} + \bigO{h^2}\\
	\label{eq:diffX2}
	\frac{\partial^2u}{\partial x_2 ^2} = \frac{u(x_1, x_2 + h) - 2u(x_1, x_2) + u(x_1, x_2 - h)}{h^2} + \bigO{h^2}
\end{align}
With remainders neglected in Equations (\ref{eq:diffX1}) and (\ref{eq:diffX2}), the Laplace operator can be approximated by adding both equations
\begin{equation}
	\label{eq:deltaU}
	\Delta u = \frac{u(x_1 + h, x_2) + u(x_1, x_2 + h) - 4u(x_1, x_2) + u(x_1 - h, x_2) + u(x_1, x_2 - h)}{h^2} + \bigO{h^2} \text{.}
 \end{equation}
Since $u$ is a four times continuously differentiable function in $\bar{\Omega}$ and the remainder is $\bigO{h^2}$, it holds $|| f_h -L_h R_h u||_{h} \in \bigO{h^2}$ as $h\rightarrow 0$, where $f_h$ is the right hand side of the BVP, $L_h$ the finite difference matrix and $R_h$ the restriction operator within $\bar{\Omega}$.
Hence, \EQ{deltaU}, also called as the 5-point stencil, is considered as a 2nd order consistent approximation of $\Delta u$.
%
% ----------------
\addSubExercise{b}
Unlike in the Exercise 1c of Assignment 3, a more practical way exists in building the $\MAT{L_h}$ in a sparse form using row \& column indices and the element value.
The command \texttt{sparse(ii, jj, aa)} in MATLAB serves exactly to this purpose, with \texttt{ii} and \texttt{jj} representing the vectors containing the row and column indices of the non-zero elements and \texttt{aa} the value of those elements respectively, or $L_h(ii(k),jj(k)) = aa(k)$ with $k$ being the integer in the range between 1 and the number of non-zero elements.
The key advantage of \texttt{sparse} is the handling of indices in an arbitrary sequence within the aforementioned vectors, such that the stencil coefficients can be written in lexicographically form one under the other at any arrangement.
%
% ----------------
\addSubExercise{c}
Please refer to online submitted \texttt{a04ex01\_Lh5.m} file.
%
% ----------------
\addSubExercise{d \& e}
The exercises d and e are combined in in the online submitted \texttt{a04ex01\_solve.m} file.
The function is called with a single input parameter \texttt{FLAG} that refers to the form of the numerical differentiation with
\begin{itemize}
	\item \texttt{FULL}: Solves the problem with full matrix,
	\item \texttt{RED}: Solves the problem with reduced differentiation matrix and accordingly modified right hand side.
\end{itemize}

\begin{figure}[p]
\vspace*{\FigUpperVSpace}
	\begin{subfigure}[t]{0.9\hsize}
		\centering
		\includegraphics[width=\linewidth]{a04ex01Laplace.png} 
		\caption{$L_h u_h = \sin{(\pi x)} \sin{(\pi y)} $}
		\label{fig:a04ex01Laplace1}
		\vspace*{8mm}
	\end{subfigure}
\par\bigskip % maximise vertical space here instead
	\begin{subfigure}[b]{0.9\hsize}
		\centering
		\includegraphics[width=\linewidth]{a04ex01Laplace2.png} 
		\caption{$L_h u_h = \sin{(\pi x + \pi/8)} \sin{(\pi y)} $}
		\label{fig:a04ex01Laplace2}
		\vspace*{5mm}
	\end{subfigure}
	\caption{Solutions of two different $L_h u_h = f(x,y)$ problems with remarks regarding the active dof number, maximum error $\max_{x,y \in \Omega_h} |u_h (x,y) - u(x,y)|$ and process duration for both flags. Since the error is same in both flags, the plot of the latter flag is omitted and the remark title is added instead.}
	\label{fig:a04ex01Laplace}
\end{figure}

The solutions of the boundary value problems $L_h u_h = \sin{(\pi x)} \sin{(\pi y)} $ and $L_h u_h = \sin{(\pi x + \pi/8)} \sin{(\pi y)} $ with homogeneous Dirichlet boundary conditions utilizing either differentiation forms was simulated with $512 \times 512$ points.
The result is illustrated in \FIG{a04ex01Laplace}, where \FIG{a04ex01Laplace1} is showcasing a zero condition on the entire boundary, and \FIG{a04ex01Laplace2} is showcasing a non-zero condition on two edges of the boundary.
It reveals that the error in both methods remain same, whereas the number of DOFs dropped significantly in the reduced method, thus nearly halving the process duration in the former problem, whereas reducing the process duration by 61\% in the latter problem.
%
% EXERCISE 2
% --------------------------------------------------------------------------------------------------------------------
\addExercise{2}{Ex2}
Given is the boundary value problem
\begin{align}
	-&au^{\prime \prime}(x) + bu^\prime(x)+cu(x) = f(x), \; x\in(0,1), \\
	&u(0)=\alpha, \; u(1)=\beta
\end{align}
for a non-uniform mesh $\bar{\Omega}_h$ with constants $a, b, c, d, \alpha, \beta \in \mathbb{R}$.
%
% ----------------
\addSubExercise{a}

The right handside $f_h$ is now defined by the restriction of f to the grid points, which now can be  non uniform, and thus we can better write $f_h=f_{x}=(f(x_1),...,f(x_N))^T$. We then have to add the vector
$$
(\alpha(a\frac{2}{h_0(h_0+h_{1})}-b\frac{-1}{h_0+h_1}),0,..,0,\beta(a\frac{2}{h_{N+1}(h_N+h_{N+1})}-b\frac{1}{h_0+h_1}))^T
$$

If we use the $D^+$ stencil the middle summand of the expression of $L_h$ turns into:

$$
b(0,\frac{-1}{h_{i+1}},\frac{1}{h_{i+1}})
$$

with right hand side a sum of $f_h$ and the boundary term:

$$
(\alpha(a\frac{2}{h_0(h_0+h_{1})}),0,..,0,\beta(a\frac{2}{h_{N+1}(h_N+h_{N+1})}-b\frac{1}{h_{N+1}}))^T
$$


If we use $D^-$:

$$
b(\frac{-1}{h_{i}},\frac{1}{h_{i}},0)
$$

with right hand side a sum of $f_h$ and the boundary term:

$$
(\alpha(a\frac{2}{h_0(h_0)}-b\frac{-1}{h_0+h_1}),0,..,0,\beta(a\frac{2}{h_{N+1}(h_N+h_{N+1})}))^T
$$


%If we 
\addSubExercise{b}

Please refer to the online submitted \texttt{a04ex02getPDE.py} file. \FIG{a04e02PDE_solve} shows the graph produced if the last section of the program is uncommented, which is the analytical solution from a03ex03 restricted to a uniform grid compared to the numerical solution on a uniform grid vector transformed by the function $T(x_i)=x_i^2$.
We can see that the two curves are quite similar to each other, and the difference grows smaller the bigger the number of grid points. 

\begin{figure}
	\centering
	\includegraphics[width=\textwidth]{a04ex02PDE_solved.png} 
	\caption{Solution of the BVP with uniform and non-uniform meshes}
	\label{fig:a04e02PDE_solve}
\end{figure}

%
% EXERCISE 3
% --------------------------------------------------------------------------------------------------------------------
\addExercise{3}{Ex3}
%
% ----------------
\addSubExercise{a}
Please refer to \texttt{a04ex03solve(eps,xh,flag)} function in the online submitted \texttt{a04ex03experiment.py} file.

%
% ----------------
\addSubExercise{b}
Please refer to \texttt{a04ex03error(eps,xh,uh)} function in the online submitted \texttt{a04ex03experiment.py} file.

%
% ----------------
\addSubExercise{c}
Please refer to \texttt{a04ex03shiskin(N, sigma)} function in the online submitted \texttt{a04ex03experiment.py} file.

%
% ----------------
\addSubExercise{d}
Please refer to online submitted \texttt{a04ex03experiment.py} file.
