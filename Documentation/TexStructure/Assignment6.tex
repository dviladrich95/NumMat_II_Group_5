\newcommand{\assignmentDate}{December 6th, 2019}

% Add title
%Institute
\begin{tabular*}{\hsize}{l@{\extracolsep{\fill}} r}
	\textsc{Technical University of Berlin}		 \hfill&								 	\\
	Faculty II - Mathematics and Natural Sciences\hfill&									\\
	Institute of Mathematics 					 \hfill&									\\
	Dr. D. Peschka, A. Selahi 		 			 \hfill&									\\
\end{tabular*}

% Title
\begin{center}
	\textbf{\Large{\courseName}}\\
	\vspace{7pt}
	\large{Homework \currentAssignment}\\
	\smallskip
	\normalsize{Submitted on \assignmentDate}
\end{center}

% Group table
\begin{center}
	\vspace{-8pt}
	\begin{tabular}{l c r}
		by \textbf{\groupNumber}		    &	 			  &		 								\\
		\hline
		\texttt{Kagan Atci} 			    & \texttt{338131} & \texttt{Physical Engineering, M.Sc.}\\
		\texttt{Navneet Singh }		 	    & \texttt{380443} & \texttt{Scientific Computing, M.Sc.}\\
		\texttt{Riccardo Parise }		    & \texttt{412524} & \texttt{Scientific Computing, M.Sc.}\\
		\texttt{Daniel V. Herrmannsdoerfer} & \texttt{412543} & \texttt{Scientific Computing, M.Sc.}\\ 
		\hline
	\end{tabular}
\end{center}

% EXERCISE 1
% --------------------------------------------------------------------------------------------------------------------
\addExercise{1}{Ex1}
Considered is the elliptic eigenvalue problem of finding an eigenfunction / eigenvector pair $(u, \lambda)$ such that $L u = \lambda u$ in $\Omega$ supplemented with suitable boundary conditions on $\partial \Omega$.
%
% ----------------
\addSubExercise{a}
Taking $L u = -u^{\prime \prime} + u$ in $\Omega = (0,1)$, the discrete eigenvalues yielded through
%
\begin{align}
	u(x) 	   &= \\
	\lambda _k &= \frac{4}{h^2} \sin{\left( \frac{h \pi k}{2}\right)} + 1
\end{align}
%
for the k-th eigenvalue.
%
In following, the eigenvalues are derived from $L_h$ using non-compact 3 point stencil with $N = 200$ and $h = 1/(N+1)$ respectively for
%
\begin{itemize}
	\item Homogeneous Dirichlet boundary conditions,
	\item Homogeneous Neumann boundary conditions,
	\item Periodic boundary conditions.
\end{itemize}
%
The list of the first 20 Eigenvalues  are stated in \TAB{a06ex01a}, while the eigenvectors of the second and the 20th eigenvalues are compared in \FIG{a06ex01a}.
%
\par
For the implementation, please refer to the online submitted \texttt{a06ex01\_solveEVPa.m} file.
\addTable{H}
		 {a06ex01a}
		 {List of the first 20 Eigenvalues ordered by magnitude.
		  2nd and 20th Eigenvalues are written in bold to be distinguished easily.}

\begin{figure}[H]
\vspace*{\FigUpperVSpace}
\def\MeshFigWidth{210pt}
	\begin{subfigure}[b]{0.5\hsize}
		\centering
		\includegraphics[width=\MeshFigWidth]{a06ex01EV2.png} 
		\caption{2nd Eigenvalue}
		\label{fig:a06ex01EV2}
	\end{subfigure}
	\begin{subfigure}[b]{0.5\hsize}
		\centering
		\includegraphics[width=\MeshFigWidth]{a06ex01EV20.png} 
		\caption{20th Eigenvalue}
		\label{fig:a06ex01EV20}
	\end{subfigure}
	\caption{Comparison of the Eigenvectors between discretized EV (black dashed), Dirichlet B.C. (red), Neumann B.C. (blue) and Periodic B.C. (green) for the 2nd and 20th Eigenvalue respectively.}
	\label{fig:a06ex01a}
\end{figure}
%
% ----------------
\addSubExercise{b}
Considering
\begin{align}
	-\Delta v &= \lambda v \;\text{ in } \Omega, \\
			v &= 0 		   \;\text{ on } \partial\Omega
\end{align}
with $\Omega = \{(x,y) \in \mathbb{R}^2 \colon \left(x-L/2\right)^2 + \left(y-L/2\right)^2 < R\}$ for given $L, R \in \mathbb{R}$ and $R>0$.
The Eigenvalues of such problem can be solved analytically utilizing the Bessel function of first kind $J_n(\xi)$ with $\xi = \sqrt{\lambda} R$ at its zero.
Taking the corresponding k-th zeros form the lecture notes, the first six Eigenvalues for $R=1$ and $R=2$  are stated in \TAB{a06ex01b}.
%
\vspace*{2\FigUpperVSpace}
\addTable{H}
		 {a06ex01b}
		 {First six Eigenvalues of the disc problem with respect to $R=1$ and $R=2$}
%
The solution was implemented in the form of a function submitted in \texttt{a06ex01\_solveEVPb.m}.
The function takes an arbitrary radius $R>0$ as input and gives the first six Eigenvalues and their $n$ values. 
%
% ----------------
\addSubExercise{c}
In this exercise, the domain $\Omega$ from assignment 5, exercise 1 has been discretized with respect to the disc problem in lecture notes with $L=2.2$ and $R=1$ and the the Laplace Operator $L_h$ has been built accordingly.
The numerical Eigenvalues have been numerically computed via reduced matrix $L_h$ for $h=1/(N+1)$ applied on three mesh resolutions with $N=7$, $N=63$, and $N=511$.
A comparison between the numerical and the exact Eigenvalues from \textbf{b)} is listed in \TAB{a06ex01c}.
The first four distinct eigenfunctions are demonstrated in \FIG{a06ex01c}.
For the implementation, please refer to online submitted \texttt{a06ex01\_solveEVPc.m} file.
%
\vspace*{2\FigUpperVSpace}
\addTable{H}
		 {a06ex01c}
		 {List of the first 6 Eigenvalues ordered by magnitude for different mesh resolution and the discretized analytical solution.}
%
\begin{figure}[H]
	\centering
	\includegraphics[width=0.9\textwidth]{a06ex01c.png} 
	\caption{Eigenfunctions of the first 4 distinct eigenvalues.
			 The value of the data has been respectively labeled with the color bar next to the plot.}
	\label{fig:a06ex01c}
\end{figure}
%
% EXERCISE 2
% --------------------------------------------------------------------------------------------------------------------
\addExercise{2}{Ex2}
%
% ----------------
\addSubExercise{a}

%
% ----------------
\addSubExercise{b}

%
% ----------------
\addSubExercise{c}

%
% ----------------
\addSubExercise{d}

% EXERCISE 3
% --------------------------------------------------------------------------------------------------------------------
\addExercise{3}{Ex3}

%
% ----------------
\addSubExercise{a}

%
% ----------------
\addSubExercise{b}

%
% ----------------
\addSubExercise{c}