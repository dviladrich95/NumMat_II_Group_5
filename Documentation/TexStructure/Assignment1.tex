\newcommand{\assignmentDate}{October 28th, 2019}

% Add title
%Institute
\begin{tabular*}{\hsize}{l@{\extracolsep{\fill}} r}
	\textsc{Technical University of Berlin}		 \hfill&								 	\\
	Faculty II - Mathematics and Natural Sciences\hfill&									\\
	Institute of Mathematics 					 \hfill&									\\
	Dr. D. Peschka, A. Selahi 		 			 \hfill&									\\
\end{tabular*}

% Title
\begin{center}
	\textbf{\Large{\courseName}}\\
	\vspace{7pt}
	\large{Homework \currentAssignment}\\
	\smallskip
	\normalsize{Submitted on \assignmentDate}
\end{center}

% Group table
\begin{center}
	\vspace{-8pt}
	\begin{tabular}{l c r}
		by \textbf{\groupNumber}		    &	 			  &		 								\\
		\hline
		\texttt{Kagan Atci} 			    & \texttt{338131} & \texttt{Physical Engineering, M.Sc.}\\
		\texttt{Navneet Singh }		 	    & \texttt{380443} & \texttt{Scientific Computing, M.Sc.}\\
		\texttt{Riccardo Parise }		    & \texttt{412524} & \texttt{Scientific Computing, M.Sc.}\\
		\texttt{Daniel V. Herrmannsdoerfer} & \texttt{412543} & \texttt{Scientific Computing, M.Sc.}\\ 
		\hline
	\end{tabular}
\end{center}

% EXERCISE 1
% --------------------------------------------------------------------------------------------------------------------
\addExercise{1}{Ex1}
\addSubExercise{a \& b}
\begin{itemize}
	\item Common PDE: Kaup-Kupershmidt equation
		\begin{equation}
			\frac{\partial u }{\partial t} = \frac{\partial ^5 u}{\partial x^5} + 10 \frac{\partial ^3 u}{\partial x^3} u + 25 \frac{\partial ^2 u}{\partial x^2} \frac{\partial u}{\partial x} + 20 u^2 \frac{\partial u}{\partial x}
		\end{equation}
		\hspace{2.4cm} is a PDE fifth order.
	
	\item Member 1 PDE: Hunter-Saxton equation
		\begin{equation}
			\frac{\partial}{\partial x} \left( \frac{\partial u }{\partial t} + u \frac{\partial u}{\partial x} \right) = \frac{1}{2} \frac{\partial ^2 u}{\partial x^2}
		\end{equation}
		\hspace{2.4cm} is a PDE second order.
	\item Member 2 PDE: Liouville equation
		\begin{equation}
			\nabla ^2 u + e ^{\lambda u} = 0
		\end{equation}
		\hspace{2.4cm} is a PDE second order.
	\item Member 3 PDE: $\varphi ^4$ - Equation
		\begin{equation}
			\frac{\partial ^2 \varphi}{\partial t ^2} - \frac{\partial ^2 \varphi}{\partial x ^2} - \varphi + \varphi ^3 = 0
		\end{equation}
		\hspace{2.4cm} is a PDE second order.
		
\end{itemize}
%
% ---------------
\addSubExercise{c}
The Navier-Stokes-Equation describes the motion of the viscous fluid substances and is expressed for compressible fluid as
\begin{equation}
	\rho(\partial_t u + u \cdot \nabla u) = - \nabla p + \mu \nabla ^2 u + f
	\label{eq:NavStokes}
\end{equation}
with $\rho$ the density, $u$ velocity vector, $p$ pressure, and $\mu$ kinematic viscosity of the fluid.
\EQ{NavStokes} is expressed in homogenous form by setting $f = 0$ as follows
\begin{equation}
	\rho(\partial_t u + u \cdot \nabla u) + \nabla p - \mu \nabla ^2 u = 0
	\label{eq:NavStokesHom}
\end{equation}
For $u (t,x) = (u_0 x_2 (H - x_2), 0)^T$ with $u_0 \in \mathbb{R}$, $x = (x_ 1, x_2) \in \Omega = \mathbb{R} \times (0, H)$, and $t \in (0, \infty)$, the partial differentiations result
\begin{align}
	\frac{\partial u}{\partial t} = (0,0)^T \label{eq:du_dt}   \\
	\nabla u = (0,0)^T  					\label{eq:grad_u} \\
	\nabla^2 u = (0,0)^T  					\label{eq:div_grad_u}
\end{align}
%
since $u$ is not $t$-dependent and $u_1$ and $u_2$ are not effected by $x_1$ and $x_2$, respectively.
Equations \ref{eq:du_dt}, \ref{eq:grad_u} and \ref{eq:div_grad_u} show that $u$ is a twice differentiable function, satisfying the homogenous Navier-Stokes PDE with a boundary condition in a domain $\Omega \in \mathbb{R}^2$, which is reffered to as the classical solution for second order PDEs.
%
For the given conditions, \EQ{NavStokesHom} can be expressed as
\begin{equation}
	\nabla p = 0
\end{equation}
This can be referred to a 2D-flow model of a fluid in a tube with a width of $H$ at any certain height, which is observed along the gravity axis. Therefore, the pressure in the domain $\Omega$ is described as $p = const. \in [0, \infty)$.
%
% EXERCISE 2
% --------------------------------------------------------------------------------------------------------------------
\addExercise{2}{Ex2}
Given is the Toeplitz-Matrix
\begin{equation}
	\MAT{K_4} = \BMAT{2 & -1 & 0 & 0\\
				-1 & 2 & -1 & 0\\
				0 & -1 & 2 & -1\\
				0 & 0 & -1 & 2} \in \mathbb{R}^{4 \times 4}
\end{equation}
and $f(x) = \frac{1}{2} \VECT{x}^T \MAT{K_4} \VECT{x}: \mathbb{R}^4 \rightarrow R$.
%
% ----------------
\addSubExercise{a}
$f(x)$ can be expressed by executing the matrix multiplication as
\begin{equation}
	f(x)= x_1^2 + (x_1 - x_2)^2 + (x_2 - x_3)^2 + (x_3 - x_4)^2 + x_4^2
	\label{eq:fx_detail}
\end{equation}
and the gradient of $f(x)$ is
\begin{equation}
	\nabla f(x) = \BMAT{\frac{\partial f}{\partial x_1} \\
					   \frac{\partial f}{\partial x_2} \\
					   \frac{\partial f}{\partial x_3} \\
					   \frac{\partial f}{\partial x_4} }					   
				=\frac{1}{2} \BMAT{4x_1 - x_2\\
								   -x_1 + 4x_2 -x_3\\
								   -x_2 +4x_3 - x_4\\
								   -x_3 + 4x_4}
				= \BMAT{2 x_1 - \frac{x_2}{2} 				  \\
					    \frac{-x_1}{2}  + 2x_2 - \frac{x_3}{2}\\
					    \frac{-x_2}{2}  + 2x_3 - \frac{x_4}{2}\\
					    \frac{-x_3}{2}  + 2x_4}
					    \label{eq:gradf}
\end{equation}
and $\MAT{K_4} \VECT{x}$ gives
\begin{equation}
	\MAT{K_4} \VECT{x} = \BMAT{2 & -1 & 0 & 0\\
			     -1 & 2 & -1 & 0\\
				  0 & -1 & 2 & -1\\
				  0 & 0 & -1 & 2}
			\BMAT{x_1\\
			      x_2\\
				  x_3\\
				  x_4}
			= 
			\BMAT{2 x_1 - \frac{x_2}{2} 				  \\
	    	    \frac{-x_1}{2}  + 2x_2 - \frac{x_3}{2}\\
			    \frac{-x_2}{2}  + 2x_3 - \frac{x_4}{2}\\
			    \frac{-x_3}{2}  + 2x_4}
			    \label{eq:k4x}
\end{equation}
Hence, the statement $\nabla f(x) = \MAT{K_4} \VECT{x}$ is verified, since \EQ{gradf} and \EQ{k4x} portray equal functionals. 
%
% -----------------
\addSubExercise{b}
A real symmetric matrix $\MAT{K_n} \in \mathbb{R}^{n \times n}$ is considered as positive definite, if $\VECT{x}^T \MAT{K_n} \VECT{x} > 0, \forall x \in \mathbb{R}^n \backslash \{0\}$. $f(x)$ is a good example for fulfilment of this condition, since the $\VECT{x}^T \MAT{K_4} \VECT{x}$ is already expanded in \EQ{fx_detail} that consists of sum of square terms, and therefore non-negative for all $x \in \mathbb{R}^n \backslash \{0\}$.
%
% -----------------
\addSubExercise{c}
In the first step of the induction $\det(K_1)$ for $n = 1$ is investigated. With
\begin{equation}
	\det(\MAT{K_1}) = \det(2) = 2
\end{equation}
the statement $det(\MAT{K_n}) = n + 1 $ is fulfilled.
\\
\\
In the second step, we prove a statement for a general n, assuming that the relation holds for every value up to n-1. The determinant $\det (\MAT{K_n})$ for $n \in \mathbb{N}$ is written with the Laplace expansion as follows
\begin{flalign}
	\nonumber
	\begin{vmatrix}
		2  & -1      &  0     & \cdots  & 0\\
		-1 &         &        &         &   \\
		   &         &        &         &   \\
		   & \ddots  & \ddots & \ddots  &   \\
		   &         &        &         &   \\
		   &         &        &         & -1\\
		0  & \cdots  &  0     &   -1    & 2
	\end{vmatrix}_n = 
	2\cdot(-1)^{n + n}
	&
	\begin{vmatrix}
		2  & -1      &  0     & \cdots  & 0\\
		-1 &         &        &         &   \\
		   &         &        &         &   \\
		   & \ddots  & \ddots & \ddots  &   \\
		   &         &        &         &   \\
		   &         &        &         & -1\\
		0  & \cdots  &  0     &   -1    & 2
	\end{vmatrix}_{n-1}
	+ \hdots \\
	\hdots -1\cdot(-1)^{n+n-1}
	&
	\begin{vmatrix}
		2  & -1      &  0     & \cdots  & 0\\
		-1 &         &        &         &   \\
		   &         &        &         &   \\
		   & \ddots  & \ddots & \ddots  &   \\
		   &         &        &         &   \\
		 0 & \cdots  &        &       0 & -1\\
	\end{vmatrix}_{n-1}
	\label{eq:laplaceExp}
\end{flalign}
The minor determinant in the second term on the right hand-side of \EQ{laplaceExp} is further expanded as
\begin{equation}
	\begin{vmatrix}
		2  & -1      &  0     & \cdots  & 0\\
		-1 &         &        &         &   \\
		   &         &        &         &   \\
		   & \ddots  & \ddots & \ddots  &   \\
		   &         &        &         &   \\
		 0 & \cdots  &        &       0 & -1\\
	\end{vmatrix}_{n-1} = 
	(-1)\cdot(-1)^{n-1+n-1}
	\begin{vmatrix}
		2  & -1      &  0     & \cdots  & 0\\
		-1 &         &        &         &   \\
		   &         &        &         &   \\
		   & \ddots  & \ddots & \ddots  &   \\
		   &         &        &         &   \\
		   &         &        &         & -1\\
		0  & \cdots  &  0     &   -1    & 2
	\end{vmatrix}_{n-2}
	\label{eq:term2exp}
\end{equation}
With \EQ{term2exp} plugged in \EQ{laplaceExp}, and assuming that stated relation holds, $\det(\MAT{K_n})$ can be expressed as
\begin{align}
	\det(\MAT{K_n}) &= 2 \det(\MAT{K_{n-1}}) - \det(\MAT{K_{n-2}})\\
					&= 2n - (n-1) = n+1
\end{align}