% !TeX spellcheck = en_GB 
\documentclass[a4paper]{article}
\usepackage{ulem}
\usepackage[T1]{fontenc}
\usepackage[latin1]{inputenc}
\usepackage{cancel}
\usepackage{ngerman}
\usepackage{amsmath}
\usepackage{mathtools}
\usepackage{graphicx}
\usepackage{geometry}
\usepackage{caption}
\usepackage{amsmath}
\usepackage{wrapfig}
\usepackage{listings}
\usepackage{color}
\usepackage{hyperref}
\usepackage{amsmath}
\usepackage{amssymb}
\usepackage[bottom]{footmisc}
\usepackage[toc,page]{appendix}

\newcommand{\R}{\mathbb{R}}

% Apply equation cross reference
\newcommand*{\EQ}[1]{Equation (\ref{eq:#1})}

\graphicspath{{./pics/}}
\geometry{
	left=25mm,
	top=20mm,
	right=20mm,
	bottom=20mm
}

\lstset{ captionpos=b,
	rulecolor=\color{black},
	breaklines=true,
	frame=single
}

\title{Group 5 NumMat II Ex1}



\begin{document}
	% % % % % % % % % % % % % 
	% Deckblatt
	% % % % % % % % % % % % %
	\begin{titlepage}
		\maketitle
		\thispagestyle{empty}
	\end{titlepage}
	\newpage


\section{Discussion and Analyze of PDEs}

\subsection{Relevant PDEs}

\subsection{Navier-Stokes}
The Navier-Stokes-Equation describes the motion of the viscous fluid substances and is expressed for compressible fluid as
\begin{equation}
	\rho(\partial_t u + u \cdot \nabla u) = - \nabla p + \mu \nabla ^2 u + f
	\label{eq:NavStokes}
\end{equation}
with $\rho$ the density, $u$ velocity vector, $p$ pressure, and $\mu$ kinematic viscosity of the fluid.
\EQ{NavStokes} is expressed in homogenous form by setting $f = 0$ as follows
\begin{equation}
	\rho(\partial_t u + u \cdot \nabla u) + \nabla p - \mu \nabla ^2 u = 0
	\label{eq:NavStokesHom}
\end{equation}
For $u (t,x) = (u_0 x_2 (H - x_2), 0)^T$ with $u_0 \in \R$, $x = (x_ 1, x_2) \in \Omega = \R \times (0, H)$, and $t \in (0, \infty)$, the partial differentiations result
\begin{align}
	\frac{\partial u}{\partial t} = (0,0)^T \label{eq:du_dt}   \\
	\nabla u = (0,0)^T  					\label{eq:grad_u} \\
	\nabla^2 u = (0,0)^T  					\label{eq:div_grad_u}
\end{align}
since $u$ is not $t$-dependent and $u_1$ and $u_2$ are not effected by $x_1$ and $x_2$, respectively.
Equations \ref{eq:du_dt}, \ref{eq:grad_u} and \ref{eq:div_grad_u} show that $u$ is a twice differentiable function, satisfying the homogenous Navier-Stokes PDE with a boundary condition in a domain $\Omega \in \R^2$, which is reffered to as the classical solution for second order PDEs.
\\ \\
For the given conditions, \EQ{NavStokesHom} can be expressed as
\begin{equation}
	\nabla p = 0
\end{equation}
This can be referred to a 2D-flow model in a tube with a width of $H$ at any certain height, which is observed along the gravity axis. Therefore, the pressure in the domain $\Omega$ is described as $p = const. \in [0, \infty)$. 
\section{Linear Algebra}

\end{document}
