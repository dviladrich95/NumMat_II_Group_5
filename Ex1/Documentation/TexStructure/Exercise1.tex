\paragraph{a) \& b)}
\begin{itemize}
	\item Common PDE: Kaup-Kupershmidt equation
		\begin{equation}
			\frac{\partial u }{\partial t} = \frac{\partial ^5 u}{\partial x^5} + 10 \frac{\partial ^3 u}{\partial x^3} u + 25 \frac{\partial ^2 u}{\partial x^2} \frac{\partial u}{\partial x} + 20 u^2 \frac{\partial u}{\partial x}
		\end{equation}
		\hspace{2.4cm} is a PDE fifth order.
	
	\item Member 1 PDE: Hunter-Saxton equation
		\begin{equation}
			\frac{\partial}{\partial x} \left( \frac{\partial u }{\partial t} + u \frac{\partial u}{\partial x} \right) = \frac{1}{2} \frac{\partial ^2 u}{\partial x^2}
		\end{equation}
		\hspace{2.4cm} is a PDE second order.
	\item Member 2 PDE: Liouville equation
		\begin{equation}
			\nabla ^2 u + e ^{\lambda u} = 0
		\end{equation}
		\hspace{2.4cm} is a PDE second order.
	\item Member 3 PDE: $\varphi ^4$ - Equation
		\begin{equation}
			\frac{\partial ^2 \varphi}{\partial t ^2} - \frac{\partial ^2 \varphi}{\partial x ^2} - \varphi + \varphi ^3 = 0
		\end{equation}
		\hspace{2.4cm} is a PDE second order.
		
\end{itemize}

\paragraph{c)}\mbox{} \\
The Navier-Stokes-Equation describes the motion of the viscous fluid substances and is expressed for compressible fluid as
\begin{equation}
	\rho(\partial_t u + u \cdot \nabla u) = - \nabla p + \mu \nabla ^2 u + f
	\label{eq:NavStokes}
\end{equation}
with $\rho$ the density, $u$ velocity vector, $p$ pressure, and $\mu$ kinematic viscosity of the fluid.
\EQ{NavStokes} is expressed in homogenous form by setting $f = 0$ as follows
\begin{equation}
	\rho(\partial_t u + u \cdot \nabla u) + \nabla p - \mu \nabla ^2 u = 0
	\label{eq:NavStokesHom}
\end{equation}
For $u (t,x) = (u_0 x_2 (H - x_2), 0)^T$ with $u_0 \in \mathbb{R}$, $x = (x_ 1, x_2) \in \Omega = \mathbb{R} \times (0, H)$, and $t \in (0, \infty)$, the partial differentiations result
\begin{align}
	\frac{\partial u}{\partial t} = (0,0)^T \label{eq:du_dt}   \\
	\nabla u = (0,0)^T  					\label{eq:grad_u} \\
	\nabla^2 u = (0,0)^T  					\label{eq:div_grad_u}
\end{align}
%
since $u$ is not $t$-dependent and $u_1$ and $u_2$ are not effected by $x_1$ and $x_2$, respectively.
Equations \ref{eq:du_dt}, \ref{eq:grad_u} and \ref{eq:div_grad_u} show that $u$ is a twice differentiable function, satisfying the homogenous Navier-Stokes PDE with a boundary condition in a domain $\Omega \in \mathbb{R}^2$, which is reffered to as the classical solution for second order PDEs.
\\ \\
For the given conditions, \EQ{NavStokesHom} can be expressed as
\begin{equation}
	\nabla p = 0
\end{equation}
This can be referred to a 2D-flow model of a fluid in a tube with a width of $H$ at any certain height, which is observed along the gravity axis. Therefore, the pressure in the domain $\Omega$ is described as $p = const. \in [0, \infty)$. 
