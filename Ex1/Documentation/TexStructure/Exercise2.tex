Given is the Toeplitz-Matrix
\begin{equation}
	\MAT{K_4} = \BMAT{2 & -1 & 0 & 0\\
				-1 & 2 & -1 & 0\\
				0 & -1 & 2 & -1\\
				0 & 0 & -1 & 2} \in \mathbb{R}^{4 \times 4}
\end{equation}
and $f(x) = \frac{1}{2} \VECT{x}^T \MAT{K_4} \VECT{x}: \mathbb{R}^4 \rightarrow R$.
%
% ----------------
\addSubExercise{a}
$f(x)$ can be expressed by executing the matrix multiplication as
\begin{equation}
	f(x)= x_1^2 + (x_1 - x_2)^2 + (x_2 - x_3)^2 + (x_3 - x_4)^2 + x_4^2
	\label{eq:fx_detail}
\end{equation}
and the gradient of $f(x)$ is
\begin{equation}
	\nabla f(x) = \BMAT{\frac{\partial f}{\partial x_1} \\
					   \frac{\partial f}{\partial x_2} \\
					   \frac{\partial f}{\partial x_3} \\
					   \frac{\partial f}{\partial x_4} }					   
				=\frac{1}{2} \BMAT{4x_1 - x_2\\
								   -x_1 + 4x_2 -x_3\\
								   -x_2 +4x_3 - x_4\\
								   -x_3 + 4x_4}
				= \BMAT{2 x_1 - \frac{x_2}{2} 				  \\
					    \frac{-x_1}{2}  + 2x_2 - \frac{x_3}{2}\\
					    \frac{-x_2}{2}  + 2x_3 - \frac{x_4}{2}\\
					    \frac{-x_3}{2}  + 2x_4}
					    \label{eq:gradf}
\end{equation}
and $\MAT{K_4} \VECT{x}$ gives
\begin{equation}
	\MAT{K_4} \VECT{x} = \BMAT{2 & -1 & 0 & 0\\
			     -1 & 2 & -1 & 0\\
				  0 & -1 & 2 & -1\\
				  0 & 0 & -1 & 2}
			\BMAT{x_1\\
			      x_2\\
				  x_3\\
				  x_4}
			= 
			\BMAT{2 x_1 - \frac{x_2}{2} 				  \\
	    	    \frac{-x_1}{2}  + 2x_2 - \frac{x_3}{2}\\
			    \frac{-x_2}{2}  + 2x_3 - \frac{x_4}{2}\\
			    \frac{-x_3}{2}  + 2x_4}
			    \label{eq:k4x}
\end{equation}
Hence, the statement $\nabla f(x) = \MAT{K_4} \VECT{x}$ is verified, since \EQ{gradf} and \EQ{k4x} portray equal functionals. 
%
% -----------------
\addSubExercise{b}
A real symmetric matrix $\MAT{K_n} \in \mathbb{R}^{n \times n}$ is considered as positive definite, if $\VECT{x}^T \MAT{K_n} \VECT{x} > 0, \forall x \in \mathbb{R}^n \backslash \{0\}$. $f(x)$ is a good example for fulfilment of this condition, since the $\VECT{x}^T \MAT{K_4} \VECT{x}$ is already expanded in \EQ{fx_detail} that consists of sum of square terms, and therefore non-negative for all $x \in \mathbb{R}^n \backslash \{0\}$.
%
% -----------------
\addSubExercise{c}
In the first step of the induction $\det(K_1)$ for $n = 1$ is investigated. With
\begin{equation}
	\det(\MAT{K_1}) = \det(2) = 2
\end{equation}
the statement $det(\MAT{K_n}) = n + 1 $ is fulfilled.
\\
\\
In the second step, we prove a statement for a general n, assuming that the relation holds for every value up to n-1. The determinant $\det (\MAT{K_n})$ for $n \in \mathbb{N}$ is written with the Laplace expansion as follows
\begin{flalign}
	\nonumber
	\begin{vmatrix}
		2  & -1      &  0     & \cdots  & 0\\
		-1 &         &        &         &   \\
		   &         &        &         &   \\
		   & \ddots  & \ddots & \ddots  &   \\
		   &         &        &         &   \\
		   &         &        &         & -1\\
		0  & \cdots  &  0     &   -1    & 2
	\end{vmatrix}_n = 
	2\cdot(-1)^{n + n}
	&
	\begin{vmatrix}
		2  & -1      &  0     & \cdots  & 0\\
		-1 &         &        &         &   \\
		   &         &        &         &   \\
		   & \ddots  & \ddots & \ddots  &   \\
		   &         &        &         &   \\
		   &         &        &         & -1\\
		0  & \cdots  &  0     &   -1    & 2
	\end{vmatrix}_{n-1}
	+ \hdots \\
	\hdots -1\cdot(-1)^{n+n-1}
	&
	\begin{vmatrix}
		2  & -1      &  0     & \cdots  & 0\\
		-1 &         &        &         &   \\
		   &         &        &         &   \\
		   & \ddots  & \ddots & \ddots  &   \\
		   &         &        &         &   \\
		 0 & \cdots  &        &       0 & -1\\
	\end{vmatrix}_{n-1}
	\label{eq:laplaceExp}
\end{flalign}
The minor determinant in the second term on the right hand-side of \EQ{laplaceExp} is further expanded as
\begin{equation}
	\begin{vmatrix}
		2  & -1      &  0     & \cdots  & 0\\
		-1 &         &        &         &   \\
		   &         &        &         &   \\
		   & \ddots  & \ddots & \ddots  &   \\
		   &         &        &         &   \\
		 0 & \cdots  &        &       0 & -1\\
	\end{vmatrix}_{n-1} = 
	(-1)\cdot(-1)^{n-1+n-1}
	\begin{vmatrix}
		2  & -1      &  0     & \cdots  & 0\\
		-1 &         &        &         &   \\
		   &         &        &         &   \\
		   & \ddots  & \ddots & \ddots  &   \\
		   &         &        &         &   \\
		   &         &        &         & -1\\
		0  & \cdots  &  0     &   -1    & 2
	\end{vmatrix}_{n-2}
	\label{eq:term2exp}
\end{equation}
With \EQ{term2exp} plugged in \EQ{laplaceExp}, and assuming that stated relation holds, $\det(\MAT{K_n})$ can be expressed as
\begin{align}
	\det(\MAT{K_n}) &= 2 \det(\MAT{K_{n-1}}) - \det(\MAT{K_{n-2}})\\
					&= 2n - (n-1) = n+1
\end{align}