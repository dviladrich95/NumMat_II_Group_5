%-------------------------
\newcommand{\addExercise}[3]
{
	\section*{Exercise #1}
	\label{sec:#2}
	\input{./TexStructure/#3}
}

% ------------------------
\newcommand{\addFramedFig}[5]
{
	\begin{figure}[#1]
		\begin{center}
			\tcbox[sharp corners,
				   boxsep=1pt,
				   boxrule=1pt, 
				   colframe=black,
				   colback=white]
				   {\includegraphics[width=#2, keepaspectratio]{#3.#4}}
			\caption{#5}
			\label{fig:#3}
		\end{center}
	\end{figure}
}
% ------------------------
\newcommand{\addChart}[4]
% #1  =
% #2  =
% #3  =
{
	\begin{figure}[#1]
		\centering
		\scalebox{#3}{\input{./Figures/FlowCharts/#2.tex}}
		\caption{#4}
		\label{fig:#2}
	\end{figure}
}
% ------------------------
\newcommand{\addTable}[3]
{
	\def\arraystretch{1.2}
	\begin{table}[#1]
		\centering
		\input{./Tables/#2}
		\caption{#3}
		\label{tab:#2}
	\end{table}
}
% ------------------------					     
\newcommand*{\mybox}[2]
{
	\begin{center}
		\colorbox{#1}{\parbox{0.9\linewidth}{#2}}
	\end{center}
}
% ------------------------
\newcommand*{\addRequirement}[1]
{%
	\mybox{gray!10}
	{%
		\begin{itemize}[partopsep=0pt]
		\itemsep2pt
		#1
		\end{itemize}
	}
}
% ------------------------
\newcommand*{\addCode}[1]
{	
	\mybox{black!05}
		  {\textsf{\footnotesize#1}}
}

% Add units to nomenclature
\newcommand{\nomunit}[1]
{
	\renewcommand{\nomentryend}{\hspace*{\fill}#1}
}

\newcommand*{\BMAT}[1]
{
	\begin{bmatrix}
		#1 
	\end{bmatrix}
}

\newcommand*{\BVEC}[1]
{
	\begin{Bmatrix}
		#1 
	\end{Bmatrix}
}

% Applying vector style
\newcommand*{\VECT}[1]
{\overline{#1}}

% Applying matrix style
\newcommand*{\MAT}[1]
{\overline{ \overline{ \mathbf{#1}}}}


% CROSS REFERENCES
%--------------------------------------------
% Apply equation cross reference
\newcommand*{\EQ}[1]{Equation (\ref{eq:#1})}

% Apply figure cross reference
\newcommand*{\FIG}[1]{Figure \ref{fig:#1}}

% Apply table cross reference
\newcommand*{\TAB}[1]{Table \ref{tab:#1}}

% Apply section cross reference
\newcommand*{\SEC}[1]{Section \ref{sec:#1}}

% Apply section cross reference
\newcommand*{\CHA}[1]{Chapter \ref{sec:#1}}

% Apply appendix cross reference
\newcommand*{\APP}[1]{Appendix \ref{sec:#1}}

% TEXT STYLE
% -------------------------------------------
% Apply xsd style
\newcommand{\XSD}[1]
{{\fontfamily{phv}\selectfont#1}}

% Apply function variable style
\newcommand*{\XSDT}[1]{\footnotesize\XSD{#1}\normalsize}

% Apply flow text
\newcommand{\FLOW}[1]
{\textbf{\fontfamily{phv}\selectfont #1}}

% Apply arrow text
\newcommand{\ARROW}[1]
{\textit{\fontfamily{phv}\selectfont #1}}

% Apply arrow in text
\newcommand{\ARROWT}[1] {\footnotesize\ARROW{#1}\normalsize}

% Apply xpath style
\newcommand*{\XPATH}[1]{\textcolor{black}{\url{#1}}}

% Apply function style
\newcommand*{\FUN}[1]{\textsf{#1}}

% Apply variable style
\newcommand*{\VAR}[1]{\texttt{#1}}

% APPENDIX
%-----------------------------------------------
\newcommand*{\AppendixResults}[2]
{
	\def\ImageWidth{0.45}
	\def\TableScale{0.5}
	\begin{landscape}
		\vfill
		\begin{table}[H]
		\noindent
			% Stress Factor Load Case 1
			\scalebox{#2}{
				\begin{subtable}{\TableScale\hsize}
					\centering
					\input{./Tables/FactorsSigmaTOP#1_LC1}
					\caption{Stress Sizing Coefficients of Load Case 1}
					\label{tab:FactorsSigmaTOP#1_LC1}
				\end{subtable}
			}
			\hfill
			% Stress Factor Load Case 2
			\scalebox{#2}{
				\begin{subtable}{\TableScale\hsize}
					\centering
					\input{./Tables/FactorsSigmaTOP#1_LC2}
					\caption{Stress Sizing Coefficients of Load Case 2}
					\label{tab:FactorsSigmaTOP#1_LC2}
				\end{subtable}
			}
			
			\noindent
			% Buckle Factor Load Case 1
			\scalebox{#2}{
				\begin{subtable}{\TableScale\hsize}
					\centering
					\input{./Tables/FactorsLambdaTOP#1_LC1}
					\caption{Stability Sizing Coefficients of Load Case 1}
					\label{tab:FactorsLambdaTOP#1_LC1}
				\end{subtable}
			}
			\hfill
			% Buckle Factor Load Case 2
			\scalebox{#2}{
				\begin{subtable}{\TableScale\hsize}
					\centering
					\input{./Tables/FactorsLambdaTOP#1_LC2}
					\caption{Stability Sizing Coefficients of Load Case 2}
					\label{tab:FactorsLambdaTOP#1_LC2}
				\end{subtable}
			}
			\caption{Optimization Results of Topology #1}
			\label{tab:SizingResults#1}
		\end{table}
		\vfill
	\end{landscape}
	% FEA Figures
	\newpage
	\begin{landscape}
		\begin{figure}[H]
			% Load Case 1
			\begin{subfigure}{\ImageWidth\hsize}
				\centering
				\includegraphics[width=\hsize]{Results/Case1/Top#1S.png} 
				\caption{Stress Distribution of Load Case 1}
				%\label{fig:TOP#1_LC1_S}
			\end{subfigure}
			\hfill
			% Load Case 2
			\begin{subfigure}{\ImageWidth\hsize}
				\centering
				\includegraphics[width=\hsize]{Results/Case2/Top#1S.png} 
				\caption{Stress Distribution of Load Case 2}
				%\label{fig:TOP#1_LC2_S}
			\end{subfigure}
			\vfill
			% Load Case 1
			\begin{subfigure}{\ImageWidth\hsize}
				\centering
				\includegraphics[width=\hsize]{Results/Case1/Top#1B.png} 
				\caption{Buckle Displacements of Load Case 1}
				%\label{fig:TOP#1_LC1_S}
			\end{subfigure}
			\hfill
			% Load Case 2
			\begin{subfigure}{\ImageWidth\hsize}
				\centering
				\includegraphics[width=\hsize]{Results/Case2/Top#1B.png} 
				\caption{Buckle Displacements of Load Case 2}
				%\label{fig:TOP#1_LC2_S}
			\end{subfigure}
			\caption{Converged \acs{fea} Results of Topology #1}
			\label{fig:TopologyResults#1}
		\end{figure}
	\end{landscape}
}

%---------------------------
%
%%%%%%%%%%%%%%%%%%%%%%%%%%%%%%%%
% LANDSCAPE SETTINGS - 4 FIGURES
%%%%%%%%%%%%%%%%%%%%%%%%%%%%%%%%
%\newpage
%\def\LandscapeHeight{175pt}
%\begin{landscape}
%	\begin{figure}[H]
%		\begin{subfigure}{0.50\hsize}
%			\centering
%			\includegraphics[height=\LandscapeHeight]{Sizing_PartMass_LC_1.png} 
%			\caption{}
%			\label{}
%		\end{subfigure}
%		\hfill
%		\begin{subfigure}{0.50\hsize}
%			\centering
%			\includegraphics[height=\LandscapeHeight]{Sizing_PartMass_LC_3.png} 
%			\caption{}
%			\label{}
%		\end{subfigure}
%		
%		\begin{subfigure}{0.50\hsize}
%			\centering
%			\includegraphics[height=\LandscapeHeight]{Sizing_PartThickness_LC_1.png} 
%			\caption{Segments}
%			\label{fig:Segments}
%		\end{subfigure}
%		\hfill
%		\begin{subfigure}{0.50\hsize}
%			\centering
%			\includegraphics[height=\LandscapeHeight]{Sizing_PartThickness_LC_3.png} 
%			\caption{Component Segments}
%			\label{fig:ComponentSegments}
%		\end{subfigure}
%		\caption{}
%		\label{}
%	\end{figure}
%\end{landscape}
%%%%%%%%%%%%%%%%%%%%%%%%%%%%%%%%%%%

%%%%%%%%%%%%%%%%%%%%%%%%%%%%%%%%%%%%
% 2 FIGURES IN PORTRAIT ORIENTATION
%%%%%%%%%%%%%%%%%%%%%%%%%%%%%%%%%%%%
%\newpage
%\begin{figure}[H]
%	\begin{subfigure}{1\hsize}
%		\centering
%		\includegraphics[width=400pt]{Sizing_PartMass_LC_1} 
%		\caption{Load Case 1}
%		\label{}
%	\end{subfigure}
%	\vfill
%	\begin{subfigure}{1\hsize}
%		\centering
%		\includegraphics[width=400pt]{Sizing_PartMass_LC_3} 
%		\caption{Load Case 2}
%		\label{}
%	\end{subfigure}
%	\caption{}
%	\label{}
%\end{figure}
%%%%%%%%%%%%%%%%%%%%%%%%%%%%%%%%%%%%

%%%%%%%%%%%%%%%%%%%%%%%%%%%%%%%%%%%%
% 2 FIGURES IN PORTRAIT ORIENTATION
%%%%%%%%%%%%%%%%%%%%%%%%%%%%%%%%%%%%
%\newpage
%\begin{figure}[H]
%	\begin{subfigure}{1\hsize}
%		\centering
%		\includegraphics[width=400pt]{Sizing_PartThickness_LC_1} 
%		\caption{Load Case 1}
%		\label{}
%	\end{subfigure}
%	\vfill
%	\begin{subfigure}{1\hsize}
%		\centering
%		\includegraphics[width=400pt]{Sizing_PartThickness_LC_3} 
%		\caption{Load Case 2}
%		\label{}
%	\end{subfigure}
%	\caption{}
%	\label{}
%\end{figure}
%%%%%%%%%%%%%%%%%%%%%%%%%%%%%%%%%%%%

%%%%%%%%%%%%%%%%%%%%%%%%%%%%%%%%%%%%
% 3 FIGURES IN PORTRAIT ORIENTATION
%%%%%%%%%%%%%%%%%%%%%%%%%%%%%%%%%%%%
%\newpage
%\begin{figure}[H]
%	\begin{subfigure}{1\hsize}
%		\centering
%		\includegraphics[width=300pt]{Sizing_EigenValue.png} 
%		\caption{Load Case 1}
%		\label{}
%	\end{subfigure}
%	\vfill
%	\begin{subfigure}{1\hsize}
%		\centering
%		\includegraphics[width=300pt]{Sizing_MaximumDisplacement.png} 
%		\caption{Load Case 2}
%		\label{}
%	\end{subfigure}
%	\vfill
%	\begin{subfigure}{1\hsize}
%		\centering
%		\includegraphics[width=300pt]{Sizing_StructuralMass.png} 
%		\caption{Load Case 3}
%		\label{}
%	\end{subfigure}
%	\caption{}
%	\label{}
%\end{figure}
%%%%%%%%%%%%%%%%%%%%%%%%%%%%%%%%%%%%

%%%%%%%%%%%%%%%%%%%%%%%%%%%%%%%%%%%%
% 3 FIGURES IN PORTRAIT ORIENTATION
%%%%%%%%%%%%%%%%%%%%%%%%%%%%%%%%%%%%
%\newpage
%\begin{figure}[H]
%	\begin{subfigure}{1\hsize}
%		\centering
%		\includegraphics[width=300pt]{PartMass_LoadCase1.png} 
%		\caption{Load Case 1}
%		\label{}
%	\end{subfigure}
%	
%	\begin{subfigure}{1\hsize}
%		\centering
%		\includegraphics[width=300pt]{PartMass_LoadCase2.png} 
%		\caption{Load Case 2}
%		\label{}
%	\end{subfigure}
%	
%	\begin{subfigure}{1\hsize}
%		\centering
%		\includegraphics[width=300pt]{PartMass_LoadCase3.png} 
%		\caption{Load Case 3}
%		\label{}
%	\end{subfigure}
%	\caption{}
%	\label{}
%\end{figure}
%%%%%%%%%%%%%%%%%%%%%%%%%%%%%%%%%%%%